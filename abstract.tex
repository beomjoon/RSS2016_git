\begin{abstract}
A desirable property of a robot is the ability to plan
high quality motions quickly across a range of planning 
scenes. Here, we propose a metric space of motion planning 
scenes based on a set of representative motions and their costs,
whose metric is defined by the structure of the cost function
associated with each scene. Based on this metric, we
propose an algorithm for generalizing a motion
from one scene to another, allowing robots to produce
better quality plans quickly as it accumulates data. 
We evaluate our algorithm in three different classes of
motion planning problems: choosing a pregrasp configuration 
for generating optimal motions, choosing 
a base location and grasp for picking an object, 
and choosing an subgoal for a standard motion 
planning algorithm. We show that, even as the 
surrounding environment changes,  we can generate 
high quality plans quickly in all these problems.
\iffalse
As the robot gathers interactions with the environment by
executing its plans and receiving feedbacks, it should 
be able to plan better quality motions more quickly 
by referring to the past plans and generalizing them to 
the given planning scene appropriately. Here we present a
plan optimization algorithm that utilizes the past search
experience for guiding the search in the new motion planning
scene. We achieve this by representing the
metric betwee planning scenes with a set of motions 
and their associated score functions, and then using this metric
to guide the search in a similar fashion to contextual bandits
algorithms. We evaluate our algorithm in three different
classes of problems, where the robot is required to optimize 
a grasp from a discrete set of grasps, across different
motion planning scenes, a base location 
in three dimensional continuous space, and eight dimensional 
subgoal for standard motion planning problem.
\fi
\end{abstract}






